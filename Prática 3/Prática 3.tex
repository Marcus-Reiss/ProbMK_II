\documentclass[a4paper, 12pt]{article}
\usepackage[top=2cm, bottom=2cm, left=2.2cm, right=2.2cm]{geometry}
\usepackage[utf8]{inputenc}
\usepackage{amsmath, amsfonts, amssymb}
\usepackage{amsthm}
\usepackage{indentfirst}
\usepackage{graphicx}
\usepackage{gensymb}
\usepackage{float}
\title{Universidade de São Paulo \\ EESC}
\author{SEM0530 - Problemas de Engenharia Mecatrônica II \\ 
Prof. Marcelo Areias Trindade \\ \\ Prática 3 - Solução de sistemas lineares
\\ \\ Aluno: Marcus Vinícius Costa Reis (12549384)}
\date{15/06/2022}

\begin{document}
	\maketitle \newpage
	
	$$\begin{bmatrix}
		k_1+k_2 & -k_2 & & & & & & & & \\ -k_2 & k_2+k_3 & -k_3 & & & & & & & \\ & -k_3 & k_3+k_4 & -k_4 & & & & & & \\
		& & -k_4 & k_4+k_5 & -k_5 & & & & & \\ & & & -k_5 & k_5+k_6 & -k_6 & & & & \\ & & & & -k_6 & k_6+k_7 & -k_7 & & & \\
		& & & & & -k_7 & k_7+k_8 & -k_8 & & \\ & & & & & & -k_8 & k_8+k_9 & -k_9 & \\ & & & & & & & -k_9 & k_9+k_{10} & 
		-k_{10} \\ & & & & & & & & -k_{10} & k_{10}
	\end{bmatrix}$$
		
	$$\begin{bmatrix}
		k_1+k_2 & -k_2 & 0 & \cdots & 0 & 0 \\ -k_2 & k_2+k_3 & -k_3 & 0 & \cdots & 0 \\ 0 & -k_3 & k_3+k_4 & -k_4 
		& \cdots & 0 \\ \vdots & \cdots & \ddots & \ddots & \cdots & \vdots \\ 0 & \cdots & -k_8 & k_8+k_9 & -k_9 & 0 
		\\ 0 & \cdots & 0 & -k_9 & k_9+k_{10} & -k_{10}	\\ 0 & 0 & \cdots & 0 & -k_{10} & k_{10}
	\end{bmatrix} \begin{bmatrix} u_1 \\ u_2 \\ u_3 \\ \vdots \\ u_8 \\ u_9 \\ u_{10} \end{bmatrix}
	=\begin{bmatrix} f_1 \\ f_2 \\ f_3 \\ \vdots \\ f_8 \\ f_9 \\ f_{10}	\end{bmatrix}$$	
	
\end{document}




