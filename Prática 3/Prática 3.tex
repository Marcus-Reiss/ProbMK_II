\documentclass[a4paper, 12pt]{article}
\usepackage[top=2cm, bottom=2cm, left=2.2cm, right=2.2cm]{geometry}
\usepackage[utf8]{inputenc}
\usepackage{amsmath, amsfonts, amssymb}
\usepackage{amsthm}
\usepackage{indentfirst}
\usepackage{graphicx}
\usepackage{gensymb}
\usepackage{float}
\usepackage{listings}
\usepackage{xcolor}
\definecolor{codegreen}{rgb}{0,0.6,0}
\definecolor{codegray}{rgb}{0.5,0.5,0.5}
\definecolor{codepurple}{rgb}{0.58,0,0.82}
\definecolor{backcolour}{rgb}{0.95,0.95,0.92}
\lstdefinestyle{mystyle}{
  backgroundcolor=\color{white},   
  commentstyle=\color{codegreen},
  keywordstyle=\color{blue},
  numberstyle=\tiny\color{codegray},
  stringstyle=\color{teal},
  basicstyle=\ttfamily\footnotesize,
  breakatwhitespace=false,         
  breaklines=true,                 
  captionpos=b,                    
  keepspaces=true,                 
  numbers=left,                    
  numbersep=5pt,                  
  showspaces=false,                
  showstringspaces=false,
  showtabs=false,                  
  tabsize=2
}
\lstset{style=mystyle}
\title{Universidade de São Paulo \\ EESC}
\author{SEM0530 - Problemas de Engenharia Mecatrônica II \\ 
Prof. Marcelo Areias Trindade \\ \\ Prática 3 - Solução de sistemas lineares
\\ \\ Aluno: Marcus Vinícius Costa Reis (12549384)}
\date{15/06/2022}

\begin{document}
	\maketitle \newpage
	
	\section{Problema}
	
	Calcular os deslocamentos de uma estrutura sujeita a carregamento de forças e/ou deslocamentos usando
	um modelo discreto de molas em série, no qual as molas tem coeficiente variável representando uma
	diminuição da área da seção transversal da estrutura.
	
	\begin{itemize}
		\item Considerando os valores de rigidez das molas, construa a matriz de rigidez do sistema tal que 
		$$\textrm{\textbf{Ku}}=\textrm{\textbf{F}},\,\,\,\textrm{\textbf{u}}= \{u_1,\,\cdots ,u_{10}\}$$
		\item Determine a solução $\textrm{\textbf{u}}= \{u_1,\,\cdots ,u_{10}\}$ para o caso no qual duas forças são aplicadas
		simultaneamente: uma de $100\,N$ na extremidade livre ($u_{10}$) e uma de $-50\,N$ na metade do comprimento ($u_5$).
		\item Determine a solução $\textrm{\textbf{u}}= \{u_1,\,\cdots ,u_{10}\}$ para o caso no qual um deslocamento 
		de $3\,cm$ é imposto à extremidade livre ($u_{10}$).
		\item Faça um gráfico ($u_n$ vs $n$) para cada condição de carregamento e mostre-os na mesma figura. 	
	\end{itemize}
	
	\textbf{Dados:}
	
	\begin{itemize}
		\item $k_n=k_{min}+\Delta ke^{-bn}$, $b=0.2$, $k_{min}=10\,kN/m$, $\Delta k=(50+0.5N)\,kN/m$, onde
		$$N=84\Rightarrow \Delta k=92\,kN/m$$		
	\end{itemize}
	
	\section{Formulações}
	
	Em primeira instância, pode-se dizer que, em virtude da aplicação de uma força externa, ou da imposição de um 
	deslocamento à extremidade livre, a estrutura em questão apresenta um perfil de deslocamento.
	
	$$\begin{bmatrix}
		k_1+k_2 & -k_2 & & & & & & & & \\ -k_2 & k_2+k_3 & -k_3 & & & & & & & \\ & -k_3 & k_3+k_4 & -k_4 & & & & & & \\
		& & -k_4 & k_4+k_5 & -k_5 & & & & & \\ & & & -k_5 & k_5+k_6 & -k_6 & & & & \\ & & & & -k_6 & k_6+k_7 & -k_7 & & & \\
		& & & & & -k_7 & k_7+k_8 & -k_8 & & \\ & & & & & & -k_8 & k_8+k_9 & -k_9 & \\ & & & & & & & -k_9 & k_9+k_{10} & 
		-k_{10} \\ & & & & & & & & -k_{10} & k_{10}
	\end{bmatrix}$$
		
	$$\begin{bmatrix}
		k_1+k_2 & -k_2 & 0 & \cdots & 0 & 0 \\ -k_2 & k_2+k_3 & -k_3 & 0 & \cdots & 0 \\ 0 & -k_3 & k_3+k_4 & -k_4 
		& \cdots & 0 \\ \vdots & \cdots & \ddots & \ddots & \cdots & \vdots \\ 0 & \cdots & -k_8 & k_8+k_9 & -k_9 & 0 
		\\ 0 & \cdots & 0 & -k_9 & k_9+k_{10} & -k_{10}	\\ 0 & 0 & \cdots & 0 & -k_{10} & k_{10}
	\end{bmatrix} \begin{bmatrix} u_1 \\ u_2 \\ u_3 \\ \vdots \\ u_8 \\ u_9 \\ u_{10} \end{bmatrix}
	=\begin{bmatrix} f_1 \\ f_2 \\ f_3 \\ \vdots \\ f_8 \\ f_9 \\ f_{10}	\end{bmatrix}$$
	
	$$\begin{bmatrix}
		k_1+k_2 & -k_2 & & & & & & & \\ -k_2 & k_2+k_3 & -k_3 & & & & & & \\ & -k_3 & k_3+k_4 & -k_4 & & & & & \\
		& & -k_4 & k_4+k_5 & -k_5 & & & & \\ & & & -k_5 & k_5+k_6 & -k_6 & & & \\ & & & & -k_6 & k_6+k_7 & -k_7 & & \\
		& & & & & -k_7 & k_7+k_8 & -k_8 & \\ & & & & & & -k_8 & k_8+k_9 & -k_9 \\ & & & & & & & -k_9 & k_9+k_{10} 
	\end{bmatrix}$$
		
	\begin{lstlisting}[language=Matlab, caption=Código utilizado para os resultados obtidos]
function u = deslocs (F)
  for i = 1:10
    kk(i) = 10 + 92*exp(-0.2*i);    
  endfor
  k = [kk 0];
  K = zeros(10);
  for i = 1:10
    for j = 1:10
      if j == i
        K(i,j) = k(i) + k(i+1);
      elseif j == i + 1
    	K(i,j) = (-1)*k(j);
    	K(j,i) = K(i,j);
      endif
    endfor
  endfor
  k
  K
  u = (K\F)';
	\end{lstlisting}
	
	
\end{document}




