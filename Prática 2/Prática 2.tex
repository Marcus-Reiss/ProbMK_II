\documentclass[a4paper, 12pt]{article}
\usepackage[top=2cm, bottom=2cm, left=2.2cm, right=2.2cm]{geometry}
\usepackage[utf8]{inputenc}
\usepackage{amsmath, amsfonts, amssymb}
\usepackage{amsthm}
\usepackage{indentfirst}
\usepackage{graphicx}
\usepackage{gensymb}
\usepackage{float}
\title{Universidade de São Paulo \\ EESC}
\author{SEM0530 - Problemas de Engenharia Mecatrônica II \\ 
Prof. Marcelo Areias Trindade \\ \\ Prática 2 - Aproximação de integrais
\\ \\ Aluno: Marcus Vinícius Costa Reis (12549384)}
\date{01/06/2022}

\begin{document}
	\maketitle \newpage
	
	\section{Problema}
	
	Um veículo se desloca com trajetória circular de raio $r=100\,m$. Considerando que ele inicia o movimento com velocidade 
	inicial de $v_0=(10+0.1N)\,m/s$ e acelera com $a_t=(4-0.01s^2)\,m/s^2$ (onde $N=84\Rightarrow v_0=18.4\,m/s$):
	
	\begin{itemize}
		\item Determine o módulo da velocidade do veículo desenvolvida ao longo da trajetória $v(s)$, faça um gráfico 
		($v$ vs $s$), e calcule a velocidade alcançada depois de percorrer $20\,m$.
		\item Determine o módulo da aceleração do veículo ao longo da trajetória $a(s)$, faça um gráfico ($a$ vs $s$), e 
		calcule a aceleração alcançada depois de percorrer $20\,m$.
		\item Usando um método numérico de aproximação de integrais, determine o tempo necessário para o veículo 
		percorrer $20\,m$.
	\end{itemize}		
	
\end{document}



